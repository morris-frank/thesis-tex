\begin{appendices}
    \section{Reparametrization of a truncated Gaussian}%
    \label{sec:appendix_truncated_gaussian}

    We propose to model the approximate posterior distribution with a truncated Gaussian. To optimize the parameters of the encoder we need to take the gradient of the samples with respect to the parameters of the distribution \(\η = \{\μ, \σ\}\). For the normal distribution, this can be cheaply done with the reparametrization trick. We do the same for the truncated distribution.

    We can take samples from the normal distribution \(\N(\μ,\σ)\) truncated to the bounds \([a,b]\) with a inverse transform sampler~\cite{murphyMachine2012}. We give an algorithm to take a sample from the truncated distribution and return its log-likelihood:

    \begin{algorithm}
        \begin{algorithmic}[1]
            \Procedure{SampleTruncatedNormal}{$ \μ,\σ,a,b $}
                \State\(a_{\t{cdf}} \gets \Φ(\÷{a - \μ}{\σ})\)
                \State\(b_{\t{cdf}} \gets \Φ(\÷{b - \μ}{\σ})\)
                \State\(\ε \sim U[a_{\t{cdf}}, b_{\t{cdf}}]\)\Comment{\(U[\·]\) gives a uniform sample.}
                \State\(\xi \gets \Φ^{-1}(\ε)\)
                \State\(x \gets \σ \· \xi + \μ\)
                \State\(ll \gets -\log{(\σ)} - \÷{1}{2}(\xi^2 + \log{(2\π)}) - \log{(b_{\t{cdf}} - a_{\t{cdf}})}\)
                \State\textbf{return} \(x,\ ll\)%
            \EndProcedure%
        \end{algorithmic}
        \caption{Sampling from a truncated Gaussian with given variance \(\σ\) and mean \(\μ\) in the bound \([a,b]\).}%
    \end{algorithm}

    The cumulative distribution function of a normal distribution and its inverse are:
    \begin{align}
        \Φ(x) &= \÷{1}{2}\·(1 + \erf(\÷{x}{\sqrt{2}}))\\
        \Φ^{-1}(x) &= {\t{erf}}^{-1}(\sqrt{2}\· ((2x)-1))
    \end{align}
    where \(\erf(\·)\) is the error function.

    Notice that we separated out the source of stochasticity to the uniform distribution. Both the error function \(\erf(\·)\) and its inverse \(\erf^{-1}(\·)\) are implemented in the PyTorch framework making the automatic computation of the gradients possible.
    \clearpage

    \section{Cross-likelihood}
    \begin{table}[!htb]
        \footnotesize%
        \begin{tabular}{lrrrr}
\toprule
{} &   drums &    bass &   other &  vocals \\
\midrule
drums  & 6.9e+00 & 6.8e+00 & 6.7e+00 & 6.9e+00 \\
bass   & 6.5e+00 & 6.8e+00 & 6.5e+00 & 6.5e+00 \\
other  & 7.8e+00 & 7.8e+00 & 7.8e+00 & 7.8e+00 \\
vocals & 6.6e+00 & 6.2e+00 & 6.6e+00 & 6.7e+00 \\
\bottomrule
\end{tabular}
%
        \caption{The mean average log-likelihood of all source channels under the each prior for the \texttt{musdb18} data.}%
        \label{tab:cross_likelihood_musdb}
    \end{table}

    \foreach\noise in {0,01,027,077,129,359}{
        \begin{table}[!htb]
            \footnotesize%
            \begin{tabular}{lrrrr}
\toprule
{} &      sin &   square &      saw &  triangle \\
\midrule
sin      &  5.3e+00 &  4.1e+00 &  4.8e+00 &   5.3e+00 \\
square   & -8.6e+00 &  1.8e+00 & -1.4e+01 &  -1.1e+01 \\
saw      &  2.5e+00 & -6.1e+00 &  3.2e+00 &   2.8e+00 \\
triangle &  4.7e+00 &  1.4e+00 &  4.0e+00 &   4.5e+00 \\
\bottomrule
\end{tabular}
%
            \caption{The mean average log-likelihood of all source channels under the each prior for the noise level \(0.\noise\).}
            \label{tab:cross_likelihood_toy_\noise}
        \end{table}
    }

    \clearpage
    \section{Noise-conditioning}
    \foreach\signal in {sin,square,saw,triangle}{
        \begin{table}[!htb]%
            \footnotesize%
            \input{graphics/noised_noised/\signal}%
            \caption{The mean average log-likelihood of the test set with added noise for the \texttt{\signal} model for each noise-conditioned density.}%
            \label{tab:noised_noised_data_\signal}
        \end{table}
    }%

\end{appendices}
