\section{Related works}
In this chapter, we discuss previous research in supervised and semi-supervised source separation.

\subsection{ICA}


\subsection{Deep Latent-Variable Models}

For our process, we have observations from the data space \({\B{x}}\∈\mathcal{D}\) for which there exists an unknown data probability distribution \(p^*(\mathcal{D})\). We collect a data set \(\{\B{x}_1\…\B{x}_N\}\) with \(N\) samples. We introduce an approximate model with density\footnote{We write density and distribution interchangeably to denote a probability function.} \(p_{\B{\θ}}(\mathcal{D})\) and model parameters \(\B{\θ}\). Learning or modeling means finding the values for \(\B{\θ}\) which will give the closest approximation of the true underlying process:

\begin{equation}
    p_{\B{\θ}}(\mathcal{D}) \approx p^*(\mathcal{D})
\end{equation}

The model \(p_{\B{\θ}}\) has to be complex enough to be able to fit the data density while little enough parameters to be learned. Every choice for the form of the model will \I{induce} biases\footnote{called \I{inductive biases}} about what density we can model, even before we maximize a learning objective using the parameters \(\B{\θ}\).

In the following described models we assume the sampled data points \(\B{x}\) to be drawn from \(\mathcal{D}\) \I{independent and identically distributed}\footnote{meaning the sample of one datum does not depend on the other data points}. Therefore we can write the data log-likelihood as:

\begin{align}
    p_{\B{\θ}}(\mathcal{D})
    &= \Π_{\B{x}\∈\mathcal{D}} p_{\B{\θ}}(\B{x})\\
    \log p_{\B{\θ}}(\mathcal{D})
    &= \Σ_{\B{x}\∈\mathcal{D}} \log p_{\B{\θ}}(\B{x})
\end{align}

The maximum likelihood estimation of our model parameters maximizes this objective.

To form a latent-variable model we introduce a \I{latent variable}\footnote{Latent variables are part of the directed graphical model but not observed.}. The data likelihood now is the marginal density of the joint latent density:

\begin{equation}
    p_{\B{\θ}}(\B{x}) = \∫ p_{\B{\θ}}(\B{x},\B{z}) d\B{z}
\end{equation}

Typically we introduce a factorization of the joint. Most commonly and simplest:

\begin{equation}
    p_{\B{\θ}}(\B{x}) = \∫ p_{\B{\θ}}(\B{x}|\B{z})p(\B{z}) d\B{z}
    \label{eq:factorized_data_likelihood}
\end{equation}

\begin{marginfigure}%
    \begin{tikzpicture}
    \node[obs]                (x) {\(\B{x}\)};
    \node[latent, left=of x]  (z) {\(\B{z}\)};

    \edge {z} {x} ; %

    \plate {xz} {(x)(z)} {\(N\)} ;
\end{tikzpicture}

    \caption{The graphical model with a introduced latent variable \(\B{z}\). Observed variables are shaded.}
    \label{fig:factorization_pgm}
\end{marginfigure}

This corresponds to the graphical model in which \(\B{z}\) is generative parent node of the observed \(\B{x}\), see~\cref{fig:factorization_pgm}. The density \(p(\B{z})\) is called the \I{prior distribution}.

If the latent is small, discrete, it might be possible to directly marginalize over it. If for example, \(\B{z}\) is a discrete random variable and the conditional \(p_{\B{\θ}}(\B{x}|\B{z})\) is a Gaussian distribution than the data model density \(p_{\B{\θ}}(\B{x})\) becomes a mixture-of-Gaussians, which we can directly estimate by maximum likelihood estimation of the data likelihood.

For more complicated models the data likelihood \(p_{\B{\θ}}(\B{x})\) as well as the model posterior \(p_{\B{\θ}}(\B{z}|\B{x})\) are intractable because of the integration over the latent \(\B{z}\) in \cref{eq:factorized_data_likelihood}.

To formalize the search for an intractable posterior into a tractable optimization problem we follow the \I{variational principle}~\cite{jordanIntroduction1999} which introduces an approximate posterior distribution \(q_{\B{\φ}}(\B{z}|\B{x})\), also called the \I{inference model}. Again the choice of the model here carries inductive biases as such that even in asymptotic expectation we can not obtain the true posterior.

Following the derivation in~\textcite[p.~20]{kingmaIntroduction2019} we introduce the inference model into the data likelihood~\footnote{The first step is valid as \(q_{\B{\θ}}\) is a valid density function and thus integrates to one.}:

test~\cite{jordanIntroduction1999}

\begin{align}
    \log p_{\B{\θ}}(\B{x})
    &= \E_{q_{\B{\θ}}(\B{z}|\B{x})} \left[ \log p_{\B{\θ}}(\B{x}) \right]\\
    &= \E_{q_{\B{\θ}}(\B{z}|\B{x})}
        \left[\log
        \÷{p_{\B{\θ}}(\B{x}, \B{z})}
          {p_{\B{\θ}}(\B{z}|\B{x})}
        \right]\\
    &= \E_{q_{\B{\θ}}(\B{z}|\B{x})}
        \left[\log
        \÷{p_{\B{\θ}}(\B{x}, \B{z})}
          {q_{\B{\φ}}(\B{z}|\B{x})}
        \÷{q_{\B{\φ}}(\B{z}|\B{x})}
          {p_{\B{\θ}}(\B{z}|\B{x})}
        \right]\\
    &= \E_{q_{\B{\θ}}(\B{z}|\B{x})}
        \left[\log
        \÷{p_{\B{\θ}}(\B{x}, \B{z})}
          {q_{\B{\φ}}(\B{z}|\B{x})}
        \right]
    +  \E_{q_{\B{\θ}}(\B{z}|\B{x})}
        \left[\log
        \÷{q_{\B{\φ}}(\B{z}|\B{x})}
          {p_{\B{\θ}}(\B{z}|\B{x})}
        \right]\\
    &= \E_{q_{\B{\θ}}(\B{z}|\B{x})}
        \left[\log
        \÷{p_{\B{\θ}}(\B{x}, \B{z})}
            {q_{\B{\φ}}(\B{z}|\B{x})}
        \right]
    +  \KL[q_{\B{\φ}}(\B{z}|\B{x}) \|
           p_{\B{\θ}}(\B{z}|\B{x})  ]
\end{align}

Note that we separated the likelihood into two parts. The second part is the (positive) Kullback-Leibler divergence of the approximate posterior from the true intractable posterior. This unknown divergence states the `correctness' of our approximation~\footnote{More specifically the divergence marries two errors of our approximate model. First, it gives the error of our posterior estimation from the true posterior, by definition of divergence. Second, it specifies the error of our complete model likelihood from the marginal likelihood. This is called the \I{tightness} of the bound.}.

The first term is the \I{variational free energy}~\todo{explain free energy} or \I{evidence lower bound} (ELBO):

\begin{align}
    \elbo_{\B{\θ}, \B{\φ}}(\B{x})
    &= \E_{q_{\B{\θ}}(\B{z}|\B{x})}
        \left[\log
        \÷{p_{\B{\θ}}(\B{x}, \B{z})}
          {q_{\B{\φ}}(\B{z}|\B{x})}
        \right]
    \label{eq:elbo}
\end{align}

We can introduce the same factorization as in~\cref{eq:factorized_data_likelihood}:

\begin{align}
    \elbo_{\B{\θ}, \B{\φ}}(\B{x})
    &= \E_{q_{\B{\θ}}(\B{z}|\B{x})}
        \left[\log
        \÷{p_{\B{\θ}}(\B{x}|\B{z}) p(\B{z})}
          {q_{\B{\φ}}(\B{z}|\B{x})}
        \right]\\
    &= \E_{q_{\B{\θ}}(\B{z}|\B{x})}
        \left[\log
        \÷{p(\B{z})}
          {q_{\B{\φ}}(\B{z}|\B{x})}
        \right]
    + \E_{q_{\B{\θ}}(\B{z}|\B{x})}
        \left[\log p_{\B{\θ}}(\B{x}|\B{z})\right]\\
    &= -\KL[q_{\B{\φ}}(\B{z}|\B{x})\|p(\B{z})]
    + \E_{q_{\B{\θ}}(\B{z}|\B{x})}
        \left[\log p_{\B{\θ}}(\B{x}|\B{z})\right]
    \label{eq:elbo}
\end{align}

Under this factorization, we separated the lower bound into two parts. First, the divergence of the approximate posterior from the latent prior distribution and second the data posterior likelihood from the latent~\footnote{this will later be the reconstruction error. How well can we return to the data density from latent space}.

The optimization of the \(\elbo_{\B{\θ}, \B{\φ}}\) allows us to jointly optimize the parameter sets \(\B{\θ}\) and \(\B{\φ}\). The gradient with respect to \(\B{\θ}\) can be estimated with an unbiased Monte Carlo estimate using data samples~\footnote{\( \∇_{\B{\θ}} \elbo_{\B{\θ}, \B{\φ}} \approxeq \∇_{\B{\θ}} \log p_{\B{\θ}} (\B{x}, \B{z}) \)}. We can \I{not} though do the same for the variational parameters \(\B{\φ}\), as the expectation of the ELBO is over the approximate posterior which depends on \(\B{\φ}\). By a change of variable of the latent variable we can make this gradient tractable, the so called \I{reparameterization trick}~\cite{kingmaAutoEncoding2014}. We express the \(z\sim q_{\B{\θ}}\) as an random sample from a unparametrized source of entropy \(\B{\ε}\) and a parametrized transformation:

\begin{equation}
    \B{z} = f_{\B{\η}}(\B{\ε})
\end{equation}

For example for a Gaussian distribution we can express \(z\sim \N(\μ,\σ)\) as \(z = \μ + \σ\·\ε\) with \(\ε\sim \N(0,1)\) and \(\η = \{\μ,\σ\}\).

\subsection{The VAE framework}

VAE~\footnotemark[\value{footnote}]\cite{rezendeStochastic2014}

The \β-VAE~\cite{higginsBetaVAE2016} extends the VAE objective with an \(\β\) hyperparameter in front of the KL divergence. The value \(\β\) gives a constraint on the latent space controlling the capacity of it. Adapting \(\β\) gives a trade-off between the reconstruction quality of the autoencoder and the simplicity of the latent representations\footnotemark[\value{footnote}]. Using such a constraint is similar to the use of in the information bottleneck~\cite{burgessUnderstanding2018}.

\subsection{Flow based models}

Another class of common deep latent models is based on \I{normalizing} flows~\cite{tabakFamily2013}. A normalizing flow is a function \(f(\B{x})\) that maps the input density to a fixed, prescribed density \(p(\ε) = p(f(\B{x}))\), in that normalizing the density~\footnote{The extreme of this idea is, of course, an infinitesimal, continuous-time flow with a velocity field.}. They use a flow for the approximate posterior \(q_{\B{\φ}}(\B{z}|\B{x})\).  Again this is commonly set to be a factorized Gaussian distribution.

For a finite normalizing flow, we consider a chain of invertible, smooth mappings.

NICE~\cite{dinhNICE2015}
- volume preserving transformations
- coupling layer
- triangular shape

Normalizing Flow~\cite{rezendeVariational2016}


RealNVP~\cite{dinhDensity2017} builds on top of NICE creating a more general, non-volume preserving, normalizing flow.

\begin{align}
    y_{1:d} &= x_{1:d}\\
    y_{d+1:D} &= x_{d+1:D} \odot \exp{(s(x_{1:d}))} + t(x_{1:d})
\end{align}

\begin{equation}
    \pf{y}{x^T} = \bM{\1_d & 0 \\ \pf{y_{d+1}:D}{x^T_{1:d}} & \diag{\left(\exp{(s(x_{1:d}))}\right)}}
\end{equation}

Glow~\cite{kingmaGlow2018} extended the RealNVP by introducing invertible 1\×1-convolutions. Instead of having fixed masks and permutations for the computations of the affine parameters in the coupling layer, Glow learns a rotation matrix which mixes the channels. After mixing the input can always be split into the same two parts for the affine transformation. Further, the authors showed that training can be helped by initializing the last layer of each affine parameter network with zeros. This ensures that at the beginning without weight update each coupling layer behaves as an identity.

\subsection{Modelling raw audio}
Deep learning models as used for image applications are unsuitable for raw audio signals (signals in \I{time-domain}). Digital audio is sampled at high sample rates commonly 16kHz up to 44kHz. The features of interest lie at scales of strongly different magnitudes. Recognizing the local-structure of a signal, like frequency and timbre, might require features at short intervals (\(\approx\) tens of milliseconds) but modeling of speech or music features happens at the scale of seconds to minutes. As such a generative model for this domain has to model at these different scales.

\begin{figure}[]
    \begin{tikzpicture}[scale=0.6, every node/.style={scale=0.6}]
    \begin{scope}
        \draw[xstep=18cm, dashed] (1,1) grid (18,5);

        \begin{scope}[xshift=0.5cm,yshift=0cm,canvas is xz plane at y=1]
            \draw[fill=blue,draw=none,fill opacity=.2]
                (0,0) sin (1,1)  cos (2,0)  sin (3,-1)  cos (4,0)
                      sin (5,1)  cos (6,0)  sin (7,-1)  cos (8,0)
                      sin (9,1)  cos (10,0) sin (11,-1) cos (12,0)
                      sin (13,1) cos (14,0) sin (15,-1) cos (16,0)
                      sin (17,1) cos (18,0);
        \end{scope}

        \foreach \x in {1,...,18} {
            \node[inner sep=.15cm, fill=blue] () at (\x,1) {};
            \node[inner sep=.15cm, fill=orange] () at (\x,5) {};
        }

        \foreach \x in {1,...,18} {
            \foreach \y in {2,...,4} {
                \node[draw=black,fill=white] (thisNode) at (\x,\y) {};
            }
        }

        \draw[decorate,decoration={brace,amplitude=4pt}]
        (0.75,1.8) -- (0.75,4.2) node [black,midway,xshift=-.4cm, rotate=90] {\small
        hidden layers};

        \draw[->] (1,0.4) -- (18.5,0.4) node[right] {time \(t\)};
    \end{scope}

    \begin{scope}[shift={(18,5)}]
        \graph [grow down, branch left=1cm, empty nodes, nodes={rectangle}, edges={draw,black}] {
            "" -- {
                "" -- {
                    "" -- { "" -- {"",""}, "" -- {"",""}},
                    "" -- { "" -- {"",""}, "" -- {"",""}}
                },
                "" -- {
                    "" -- { "" -- {"",""}, "" -- {"",""}},
                    "" -- { "" -- {"",""}, "" -- {"",""}}
                }
            }
        };
    \end{scope}
\end{tikzpicture}

    \caption{An example of how dilated convolutions are used in the WaveNet. We see three hidden layers with each a kernel size of two. By using the dilations the prediction of the new output element has a receptive field of 18. This convolution is \I{causal} as the prediction depends only on previous input values. Causality is enforced through asymmetric padding.}
    \label{fig:wavenet}
\end{figure}

The \B{WaveNet}~\cite{vandenoordWaveNet2016} introduced an autoregressive generative model for raw audio. It is build upon the similar PixelCNN~\cite[\protect\label{pixelcnn}]{vandenoordConditional2016} but adapted for the audio domain.  The WaveNet accomplishes this by using dilated causal convolutions~\footnote{The \I{à trous} alogrithm (\textcite{holschneiderRealTime1990}), a common tool in signal processing, uses a Wavelet kernel that is dilated to multiple scales. A dilated convolution, first used in \textcite{yuMultiScale2016} differs in that the convolution operator it-self is \I{dilated}, having an internal stride.}.  Using a stack of dilated convolutions increases the receptive field of the deep features without increasing the computational complexity, see~\cref{fig:wavenet}. Further, the convolutions are gated and the output is constructed from skip connections. For the sturcture of a hidden layer refer to \cref{fig:wavenet_layer}. A gated feature, as known from the LSTM~\cite{hochreiterLong1997a}, computes two outputs: one put through an sigmoid \(\σ(\·)\) activation and one through an \(\tanh(\·)\) activation. The idea being that the sigmoid (with an output range of \([0, 1]\)) regulates the amount of information, thereby gating it, while the \(\tanh\) (with a range of \([-1,1]\)) gives the magnitude of the feature. The output of the WaveNet is the sum of outflowing skip connections added after each (gated) hidden convolution. This helps fusing information from multiple time-scales (\I{low-level} to \I{high-level}) and makes training easier~\cite{szegedyGoing2015}. The original authors tested the model on multiple audio generation tasks. They used a \μ-law encoding~\cite{Recommendation1988} which discretizes the range \([-1, 1]\) to allow a set of \μ targets and an multi-class cross-entropy training objective. While being quite unnatural this is done to avoid making any assumptions about the target distribution. Sound generation with a WaveNet is slow as the autoregressiveness requires the generation value by value\footnote{Why are they doing that then? The WaveNet setting is generating waveforms, by giving the previously generated values as the input and conditioning the process on target classes, \(p(x_t|x_{1:t},c_t)\). Therefore the generation has to happen value-by-value.}. This can be alleviated by keeping intermediate hidden activations cached~\cite{paineFast2016}. The WaveNet can be conditioned by adding the weighted conditionals in the gate and feature activations of the gated convolutions (see~\textcite{vandenoordConditional2016}).

\begin{marginfigure}
    \begin{tikzpicture}
    \matrix[]
    {
        & & \node (g)  {g}; &
        \node (gs) {\(\σ\)}; & & & & &\\

        \node (featold) {}; &
        \node (dilate) {dilate}; & & &
        \node (times)  {\(\odot\)}; & &
        \node (chfeat) {\(1\× 1\)}; &
        \node (plfeat) {+}; &
        \node (featnew) {};\\

        & & \node (f) {f}; &
        \node (ft)  {\(\tanh\)}; & &
        \node (chskip) {\(1\× 1\)}; & & &\\

        \node (skipold) {}; & & & & &
        \node (plskip) {+}; & & &
        \node (skipnew) {};\\
    };
    \begin{scope}[every path/.style={draw, ->}]
        \path (featold) -- (dilate);
        \path (dilate) |- (g);
        \path (dilate) |- (f);
        \path (g) -- (gs);
        \path (f) -- (ft);
        \path (gs) -| (times);
        \path (ft) -| (times);
        \path (times) -- (chfeat);
        \path (chfeat) -- (plfeat) -- (featnew);
        \path (times) -| (chskip);
        \path (chskip) -- (plskip);

        \path (dilate) |- (1,1) -| (plfeat);

        \path (skipold) -- node [near start] {skip} (plskip) -- (skipnew);
    \end{scope}
\end{tikzpicture}

    \caption{Hidden layer as in the WaveNet. Infomration flows from left, gets dilated and through the gate and filter. The result gets added to the skip flow and the hidden feature, each with a channel mixer before.}%
    \label{fig:wavenet_layer}
\end{marginfigure}

% INTERANL TODO: explain PixelCNN++ even if I don't use it know? %

The first generative model for sound using WaveNet is \B{NSynth}~\cite{kalchbrennerEfficient2018}. They construct a VAE where encoder and decoder are both WaveNet-like. The, so-called, \I{non-causal temporal encoder} uses a stack of dilated residual non-causal convolutions. The convolutions are not gated and no skip-connections are used. The decoder is a WaveNet taking the original input chunk as an input and predicitng the next value, while being conditioned on the latent variable. The authors use this model to learn latent temporal codes from a new large set of notes played by natural and synthesized instruments. The latent of the VAE is conditioned on the pitch of these notes. While the model is difficult to train the show great improvement of the WaveNet base VAE compared to a spectral-based VAE.

\textcite{chorowskiUnsupervised2019} presents another WaveNet-based VAE. They are learning speech representations, unsupervised. The encoder is a residual convnet and takes the MFCC of the signal as its input. As the bottleneck they found a VQ codebook~\cite{vandenoordNeural2017} to be most successfull. The decoder is an autoregressive WaveNet conditioned on the latent features. With this model they find it to perform best when basing the latent features on the mel-cepstra but reconstructing the signal with the WaveNet in time-domain.

In~\cite{prengerWaveGlow2018}

FloWaveNet~\cite{kimFloWaveNet2019a}

\subsection{Source separation}
All here presented model maximize \(p(\B{s}|\B{m})\) for one source (e.g. extracting only the singing voice out of a mix) or \(p(\B(s)_1,\…,\B{s}_N|\B{m})\) for multiple sources. Note that in the second case the conditional likelihood is not factorized, meaning we build a shared model for all sources.

\textcite{rethageWavenet2018} use a WaveNet for speech denoising. Speech denoising is a special case of source separation as the observed mix is separated into true signal and noise. The authors made the WaveNet non-causal by making the padding symmetrical. Further they used \(L_1\)-loss, instead of the multi-class \μ-law objective. They show that for this case the real valued predicitons bevae better.

\textcite{janssonSinging2017} were the first to use an U-Net architecture (see~\cref{fig:unet}) for musical source separation. They used a convolutional U-Net on spectrograms of musical data to extract the singing voice. Input to the network is the spectral magnitude and the output prediction is an equally-sized masked. The voice signal reconstruction is then done by multiplying the input magnitude with the predicted mask and using the original phase information from the mix, unaltered. The training objective is the \(L_1\) loss between the masked input spectrogram and the target signal.

\begin{marginfigure}
    \begin{tikzpicture}
    \matrix [column sep=0mm, row sep=4mm, every node/.style={rectangle, draw, align=center, text width=4em, fill=white}]
    {
        \node[fill=blue, draw=white, text=white] (input)   {input}; & &
        \begin{scope}
            \node[fill=orange, draw=white, text=white] (output1)  {output};
            \foreach \X [count=\Y] in {2,3,4}
            {\node[fill=orange,draw=white,text=white,anchor=north west,below right=0.5mm and 0.5mm of output\Y.north west] (output\X){output};}
        \end{scope}\\

        \node (ds1)     {conv}; & &
        \node (us1)     {conv};\\

        \node[draw=none] (ds)      {\(\…\)}; & &
        \node[draw=none] (us)      {\(\…\)};\\

        \node (ds2)     {conv}; & &
        \node (us2)     {conv};\\

        & \node (fuse)  {bottleneck}; &\\
    };
    \begin{scope}[every path/.style={draw, -latex'}]
        \path         (input) -- (ds1);
        \path         (ds1)   -- (ds);
        \path         (ds)    -- (ds2);
        \path         (ds2)   -- (fuse);
        \path         (fuse)  -- (us2);
        \path         (us2)   -- (us);
        \path         (us)    -- (us1);
        \path         (us1)   -- (output4);
        \path[dashed] (ds1) -- node[label=cond] {} (us1);
        \path[dashed] (ds2) -- node[label=cond] {} (us2);
    \end{scope}
\end{tikzpicture}
%
    \caption{The U-Net}%
    \label{fig:unet}
\end{marginfigure}

The Wave-U-Net\cite{stollerWaveUNet2018} brings this idea into the time-domain. The downstreaming and upstreaming layers are replaced with WaveNet style 1D convolutional layers. Further here the model is predicting multiple source signals.

\textcite{lluisEndtoend2019} adapted a time-domain WaveNet for musical source separation. The network is non-causal~\footnote{In this setting the WaveNet can be non-causal because the prediction happens from the given mix and is not autoregressive.} The WaveNet directly outputs the separated sources and is trained fully supervised. They show this simple setup performing well against spectrogram based baselines. Also, the experiments show a higher performance with a network that is deeper but less wide~\footnote{Deepness of a network refers to the number of hidden layers. Wideness refers to the number of kernels/features of each of these hidden layers. Making the WaveNet deeper significantly increases its receptive field.}

Demucs~\cite{defossezDemucs2019} is another extension building on the U-Net idea. Having a similar structure to the Wave-U-Net they introduce a bidirectional LSTM at the bottleneck~\cite{defossezSING2018}. The LSTM is responsible for keeping long-term temporal informational by running over the high-level latent along the time dimension. They can outperform spectrogram based approaches.
