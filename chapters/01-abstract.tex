\chapter{Abstract}%
\label{ch:abstract}%
In this research, we introducing a new approach to musical source separation. We are proposing to approximate the generative model of the observed mixed sound signals. To be able to build a model of the generation, we build a separate independent prior model for each source signal, that is contained in the mixed signals. We then propose to find the correct source signals with either a variational modeling approach or through direct sampling from the prior models. In the variational approach, we build approximate posteriors of the true sources given the mix with the previously trained source models acting as the prior. In the sampling approach, we extract the source signals directly from the prior models through an iterative constrained Langevin sampling chain. We state the conditions and difficulties of both approaches, theoretically.

We find that, using modern normalizing flow architectures and optimization regimes for the prior models we are not able to construct generative models that fulfill the needed qualities to be used in this approach. We show that we are not able to optimize the prior models to be smooth and discriminative against out-of-distribution samples at the same time. This reinforces previous research that has hinted, that contemporary generative flow models behave unexpectedly outside the close bounds of their training data distribution. Thereby this work sends another warning to make assumptions about the learned distributions of these models, especially in the audio domain.
