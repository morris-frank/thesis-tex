\chapter{Conlusion}
In this research work, we introduce a new approach to unsupervised source separation. We describe theoretically that source separation can be achieved by modeling the generative process of the mixed-signal from which then likely latent sources are extracted. Because we propose to model the data distribution of each signal source independently we remove the necessity of having mixed-unmixed training pairs. We sketch out two approaches to retrieve samples from the posterior of these sources, either by learning separate approximate posteriors or by directly sampling from the prior density under the mixing constraint.

Our experiments reinforce a suspicion that was also experimentally found in prior work under different circumstances. Current deep generative models do not learn a density that is discriminative against out-of-distribution samples. We show that in our case the generative priors loose their ability to detect out-of-distribution samples when trained with additive noise which is added to smooth the learned densities. Therefore our work further demonstrates to be cautious when applying current flow-based models to data outside close bounds of their training distribution.

Because of these difficulties we were not able to practically implement the presented approach to musical source separation. Nevertheless, further work on constructing generative sound models that generalize but also discriminate can be brought into the framework of the method to achieve unsupervised musical source separation.
