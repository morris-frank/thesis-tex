\chapter{Research Question}%
\label{ch:question}%

Source separation is the task of finding the set of source signals \(\s = {[\s_1,\…\s_k,\…,\s_N]}^T \∈ \ℝ^{N\× T}\) to an observed mix of those sources \(\m\∈\ℝ^{1\× T}\). The mix is generated from the source signals with a mixing function \(\m = f(\s)\). The task is to find an inverse model \(g(\·)\) which retrieves \(\s\):
\begin{align}
    \m = f(\s)\\
    g(\m) \approxeq \s
\end{align}

In the case of musical source separation, the sources are different instruments~\footnote{Think of the sources being \{\I{guitar},\I{piano},\I{voice},\…\}}. The mix signal is the combined recording of the instruments. Differnt instruments can make similar sounds, making the reversal of the mixing operation strongly unidentifiable and generally impossible.\medskip

Imagine a set of songs played by the three instruments guitar, piano and drums. To make the approximation of the inverse model easier multiple sources of supervision can be supplied: First, by supplying the number of sources \(N\) in the mix. In the example that would be knowing that each song contains three instruments. Second, by supplying the source classes for each contained source. In the example we would supply the three classes guitar, piano and drums. These two labels are not the same, as in the first case a subset of the sources could be played by the same instrument. Lastly we can supervise the modeling, by actually providing tuples \((\m, \s)\) of associated mixes and sources. In the example case this means, having the single recordings each of the three instruments and the mixed recording of the song. This supervision is the most costly one, as there exists no large and public source of such data and source signals cannot be retrieved through manual labeling.

When we refer to our approach as unsupervised we mean this costly form of supervision. We want to learn the source separation without using tuples of the form \((\m, \s)\) during training.\medskip

More precisely our research question is, whether we can train such a separation model by training separate generative models for each source instrument. We want to use these prior models to extract the signals from the mix that are close to the distributions of the prior models. We investigate whether it is possible to train such a prior model from data of one source~\footnote{e.g.\ only guitar sounds} and be able use this model find guitar sounds in mixed songs.
